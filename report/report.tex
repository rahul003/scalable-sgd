\documentclass[a4paper,11pt]{article}
\usepackage[T1]{fontenc}
\usepackage[utf8]{inputenc}
\usepackage{lmodern}
\usepackage{palatino}
\usepackage{subcaption}
\usepackage{graphicx, wrapfig}
\usepackage{multirow}
\title{SDS 394C Final project report\\{\Large Parallel stochastic gradient descent for matrix factorization}}
\author{Rahul R Huilgol, rrh2226}

\begin{document}

\maketitle
\tableofcontents

\begin{abstract}
In this report, I present the results of my project on parallelizing Stochastic gradient descent for Matrix factorization using Openmp. 
\end{abstract}

\section{Motivation}
Largely because of the internet, today has become the age where everyone is flooded with data. Companies have loads of data to analyze and users are inundated with choices making it hard for them to find what they are looking for. This has led to the areas of Information retrieval and information filtering become very important. Those companies which can efficiently mine their data to recommend the right information to users will be able to hold the interests of the user. Recommender systems have great applications in retail and entertainment sectors especially. Companies like Amazon and Netflix are trying to optimize their recommendation algorithms and utilize as much data as they can. This large amount of data calls for efficient ways of processing data, leading to parallel and distributed algorithms.

One popular class of recommender systems are based on the idea of collaborative filtering. This approach to recommendation does not rely on information about the product, rather on the similarity between different users, and on the past ratings of the user. Consider the case of a company wanting to recommend movies to users based on their past reviews of movies. The basic idea on which these systems work on is the following. Suppose a user A has rated a movie P highly, and user B has also rated movie P highly. If user B has rated another movie Q highly, then its likely that user A will also like the movie Q because they showed similar interests based on their past ratings. To formalize this argument and domain, let there be a matrix which stores the ratings of all movies by all users. The dimensions of this matrix will obviously be (number of users x number of movies). Like Fig. \ref{fig:matrix} shows, typically only a few entries of the matrix are available, because clearly all users can't be expected to rate all existing movies.

\begin{figure}[h]
   \centering
    \includegraphics[width=0.75\linewidth]{matrix.png}
    \caption{Example partially filled users x movies matrix}
   \label{fig:matrix}
\end{figure}

The task then reduces to a problem called Matrix Completion, i.e. to fill this matrix with estimated ratings. These estimated ratings can then be used for recommendations.
\section{Approach}
The intuition behind apporaching this problem is that there should be some latent(hidden) features that determine how a user rates an item. We can try to determine these latent features using the data we have, and then use the latent features to make predictions. This approach to Matrix completion can be formulated as factorizing the large matrix say $V$ into two smaller matrices $W$ and $H$. 
\section{Conclusion}
\begin{thebibliography}{9}
  \bibitem{dsgd}
  

\end{thebibliography}
\end{document}
